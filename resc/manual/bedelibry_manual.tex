
\documentclass{book}

\usepackage[paperwidth=6in, paperheight=9in]{geometry}
\usepackage{epigraph}
\usepackage{graphics}
\usepackage{listings}
\usepackage{xcolor}
\usepackage{fancyvrb}

% Japanese language support
\usepackage{xeCJK}
\setCJKmainfont{IPAMincho}
\setCJKsansfont{IPAGothic}
\setCJKmonofont{IPAGothic}

\definecolor{predColor}{RGB}{36, 64, 242}
\definecolor{varColor}{RGB}{255, 155, 79}
\definecolor{reservedColor}{RGB}{161, 24, 158}
\definecolor{entityColor}{RGB}{122, 124, 133}
\definecolor{arityColor}{RGB}{122, 124, 133}
\definecolor{trueColor}{RGB}{7, 130, 23}
\definecolor{falseColor}{RGB}{255, 28, 28}
\definecolor{noSolutionColor}{RGB}{255, 198, 28}
\definecolor{typeColor}{RGB}{134, 27, 245}
\definecolor{constrColor}{RGB}{122, 124, 133}
\definecolor{stringColor}{RGB}{122, 124, 133}
\definecolor{commentColor}{RGB}{77, 77, 77}
\definecolor{moduleColor}{RGB}{36, 64, 242}

\definecolor{codegreen}{rgb}{0,0.6,0}
\definecolor{codegray}{rgb}{0.5,0.5,0.5}
\definecolor{codeblue}{rgb}{0,0.15,0.82}
\definecolor{backcolour}{rgb}{1,1,1}

\lstdefinestyle{mystyle}{
    backgroundcolor=\color{backcolour},   
    commentstyle=\color{codegray},
    keywordstyle=\color{codeblue},
    numberstyle=\tiny\color{codegray},
    stringstyle=\color{codegreen},
    basicstyle=\ttfamily\footnotesize,
    breakatwhitespace=false,         
    breaklines=true,                 
    captionpos=b,                    
    keepspaces=true,                 
    numbersep=5pt,                  
    showspaces=false,                
    showstringspaces=false,
}

\usepackage{etoolbox}
\makeatletter
\patchcmd{\@verbatim}
  {\verbatim@font}
  {\verbatim@font\scriptsize}
  {}{}
\makeatother

\author{Nathan Bedell}
\title{Bedelibry Prolog: An introduction}

\setlength\epigraphwidth{.8\textwidth}
\setlength\epigraphrule{0pt}

% Copyright
\usepackage[type={CC},modifier={by-nc-sa},version={3.0}]{doclicense}
\usepackage[hidelinks]{hyperref}

\newenvironment{dedication}
{\clearpage           % we want a new page
	\thispagestyle{empty}% no header and footer
	\vspace*{\stretch{1}}% some space at the top
	\itshape             % the text is in italics
	\raggedleft          % flush to the right margin
}
{\par % end the paragraph
	\vspace{\stretch{3}} % space at bottom is three times that at the top
	\clearpage           % finish off the page
}

\lstdefinelanguage{BliProlog}{
    morekeywords={rel, proc, constructor, datatype, type},
    sensitive=true, % keywords are case-sensitive
    morecomment=[l]{%}, % l is for line comment
    morecomment=[s]{/*}{*/}, % s is for start and end delimiter
    morestring=[b]" % defines that strings are enclosed in double quotes
}

\begin{document}

\maketitle

% copyright page
\pagebreak
\pagenumbering{gobble}
\begin{center}
	$\;$ \\ $\;$ \\ $\;$ \\ $\;$ \\
  	$\;$ \\ $\;$ \\ $\;$ \\ $\;$ \\
	$\;$ \\ $\;$ \\ $\;$ \\ $\;$ \\
  	$\;$ \\ $\;$ \\ $\;$ \\ $\;$ \\ $\;$ \\ $\;$ \\
	\doclicenseThis
% \textcopyright \, Nathan Bedell, 2019 -- except where otherwise specified. All rights reserved.
\end{center}

% dedication
%\pagebreak
%	\begin{dedication}
%		To my friends, family, and loved ones -- all that have \\ stood by me and believed that I could make it this far.
%	\end{dedication}
%\pagebreak

\chapter*{Preface}

\epigraph{Concepts lead us to make investigations. They are the expression of our interest and direct our interest.}{Wittgenstein}

Ineviatably, the question that one has to answer when implementing or designing a new programming language is: \textit{Why?}. We already have Haskell, Scala, Prolog, Scheme, Clojure, Java, Kotlin, C\#, F\#, Nim, Idris, Agda, Isabelle/Hol, Elm, ReasonML, Javascript -- not to mention some of the more obscure offerings out there like Hume, Curry, Mercury, Ceylon, Caledon, Twelf, Flora-2 -- and thousands of others. Why do we need a new programming language?

Initially, Bedelibry Prolog was developed to be used as a query and schema definiton language for \textit{Bedelibry} -- am open-source library and knowledge representation framework oriented towards helping working researchers organize their notes, references, and thoughts on the topics that they are interested in by allowing the user to make \textit{ontologies} for different subjects, and making \textit{semantic annotations} in their notes in a formal language to aid the future retrieval of facts relevant to their research via Bedelibry's query language. And thus, the language was designed due to the bad fit of existing solutions.

The first choice for the query language of Bedelibry was Prolog -- as one of the requirements of the query language for the project was that it must support \textit{deductive rules} that allow making inferences from different basic facts. SQL alone would not do. I wanted to be able to make queries that involve complicated logical inferences, not simply queries that involve retrieving data from a store. Prolog also had the advantage that it's syntax was clean, elegant, and familiar -- being similar enough to the language of propositional logic.

However, one of the issues with Prolog is its lack of a type system. One of the design requirements of Bedelibry is that it allows the user to declare \textit{ontologies} -- and it should be easy to check that a given prolog file typechecks against an ontology. Of course, the idea of a "Prolog with types" is not new. This requirement led me to consider existing solutions such as Twelf, Lambda Prolog, HiLog, HOPES, Issabelle/HOL, Mercury, Curry, Caledon, and Flora-2. 

However, these projects either did not meet my needs by being too focused on \textit{proof search} and \textit{formal verification} rather than on knowledge representation, being unmaintained, and/or difficult to install and distribute, or by moving too far away from the simple Prolog-like syntax that I wanted to support.

In addition, in doing my own research into logic programming, and playing around with Prolog itself, I discovered that Prolog had some features that I did not want to support. The default depth first search and the ubiquitous ``cut'' feature made it difficult to work with Prolog as a query engine for a general-purpose query engine for a knowledge base with facts and rules (in my opinion, where logic programming really shines!). In short, I did not find Prolog's approach to combining logical and imperative features very satisfying. This led me to consider the excellent ``pure Prolog'' implementation by Ken Friis Larsen, which supported a breadth first search strategy, and being ``pure'', did not support any of Prolog's imperative features that I was not fond of. In addition, Ken Friis Larsen's implementation was in \textit{Haskell} -- a language I am very familiar with and fond of.

And so, I started hacking away! I added a proper REPL, some new syntatic features of my own creation, a type system, modules, a compiler -- and this all came together to create the new programming language that I call Bedelibry Prolog. 

Bedelibry Prolog was created with it's application to Bedelibry in mind -- the interpreter itself can be run in server mode to provide a REST API that the rest of the Bedelibry ecosystem can interact with. Relations can be labeled ``stored'' to indicate that queries involving these relations involve facts that might be stored in the central bedelibry server (which uses a SQL backend for persistance). Thus, the language itself was concieved in a similar vein to Emacs Lisp or Guile Scheme -- general purpose programming languages, but tied to specific application(s), and with features which makes integration into those applications smooth and simple. 

However, after it's conception as a typed Prolog with application-specific features tied to Bedelibry, Bedelibry Prolog eventually became a more general experiment in programming language design. I wanted to take the features of pure logic programming I liked, and combine them with what I saw as the good parts of other programming paradigms. I wanted good support for algebraic data types like in Haskell, but with more flexibility -- similar to the OOP paradigm, but without what I see as the mistakes of enterprise OOP languages such as C\# and Java. I wanted "pure-by-default" functional programming like in Haskell, with "escape hatches" like using \textit{StateT} and \textit{IO}, but with more flexibility, and rather than the implementation of imperative features in Prolog. 

And so, Bedelibry Prolog was born. Somewhere in-between statically typed functional programming a la Haskell, logic programming, and object-oriented programming a la Scala -- what, to me, seems like a hole in the design space of multi-paradigm languages that needed to be filled. A new paradigm inbetween the cracks of the old, with \textit{types}, \textit{relations}, \textit{procedures}, \textit{entities}, and \textit{rules} as the main building blocks.

So, why bother? For one thing, I think this project came to the point that it did out of pure curiosity on my part. Since I first learned how to code in Python, and I was exposed to the idea of \textit{functional programming}, I have always been interested in how different programming paradigms allow the user to express ideas more succinctly, and at the appropriate level of abstraction. New concepts, and new paradigms can be powerful in this way. And thus, I hope that in this book, I can convince the reader that the concepts in Bedelibry Prolog are worthwhile, and can make them into better programmers. 

\tableofcontents

\chapter{Introduction}

\section{Organization of this book}

This book is intended to be both an introduction, and documentation for the Bedelibry Prolog programming language. In addition, this book also contains signifigant exposition on the inspiration behind the language desing of Bedelibry Prolog, as well as case studies for some potential uses cases for which Bedelibry Prolog would be particularly well-suited. Thus, not everyone will want to read this entire reference linearly. Some sections/chapters are optional.



\section{Getting started with Bedelibry Prolog}

To get up and running with Bedelibry Prolog, you must have a working installation of the appropriate version of the Haskell Platform on your machine. Bedelibry Prolog is currently built and tested using ghc version 8.4.4. To get Bedelibry Prolog up and running on recent version of Ubuntu Linux is as simple as running

\begin{verbatim}
  > sudo add-apt-repository -y ppa:hvr/ghc
  > sudo apt install ghc-8.4.4
  > sudo apt install cabal-install
\end{verbatim}

\noindent to install the appropriate version of the Haskell platform, and then running

\begin{verbatim}
  > git clone https://github.com/Sintrastes/bli-prolog
  > cd bli-prolog
  > cabal install
\end{verbatim}

\noindent to install the Bedelibry Prolog interpreter \textit{blipl} and compiler \textit{blic}. Passing a command-line argument of \textit{--help} to either of these commands will yield the appropriate documentation on the command-line arguments to these commands. We will display these here for reference:

\begin{verbatim}
> blipl --help                                                                                                                                                                                           127 ↵ ✹ ✭
bli-prolog interpreter v0.3.0, (C) Nathan Bedell 2019

options [OPTIONS] [GOALSTRING]

Common flags:
     --search'=SEARCH       Specify wether to use DFS, BFS, or Limited
     --program'=FILE        Prolog file with clauses
     --schema'=ITEM         Schema file
  -l --limit'=INT           Limit the number of solutions found
  -d --depth'=INT           Maximum depth to traverse when using limited
                            search
  -v --verbose'             Specify whether or not to use verbose output (on
                            by default)
  -n --nocolor'             Turn off colors in the REPL.
  -j --json'                Specify whether or not json output formatting is
                            used for queries.
     --server'              Starts a REST server for processing bli prolog
                            queries if set.
     --bedelibrymode'=ITEM  Sets the mode of interaction between bli-prolog
                            and the bedebliry server.
     --port'=INT            Port number to start the server.
     --burl'=ITEM           URL of the bedelibry server configured to work
                            with bli-prolog.
  -? --help                 Display help message
  -V --version              Print version information
     --numeric-version      Print just the version number
\end{verbatim}

\begin{verbatim}
> blic --help                                                                                                                                                                                                  ✹ ✭
bli-prolog compiler v0.3.0, (C) Nathan Bedell 2019.
   Visit https://github.com/Sintrastes/bli-prolog for more documentation on Bedelibry Prolog.

blic [source_file]                   Compiles the specified source file to an executable
                                     in the same directory, with the same name as the source
                                     file, but without extension.
blic [source_file] [output_file]     Compiles the specified source file to an exectuable
                                     [output_file].

Flags:
  --help            Displays this help screen.
  --dyn             Compiles the file, dynamically linknig any dependent modules.
  --bytecode        Compiles to Bedelibry Prolog bytecode instead of linking into
                    a static executable.
  --no-typecheck    Skips the typechecking step.
  --only-typecheck  Only typecheck the given file, skipping the compilation step.
\end{verbatim}

If you are interested in Bedelibry more generally, and not just Bedelibry Prolog, you will also want to install the \textit{bli} command-line tool, and and instance of the Bedelibry server. 

\chapter{Logic Programming}

\epigraph{One person might say "A proposition is the most ordinary thing in the world", and another, "A proposition -- that's something very remarkable!" -- And the latter is unable simply to look and see how propositions work. For the forms of the expressions we use in talking about propositions and thought stand in his way.}{Wittgenstein}

Bedelibry Prolog is first and foremost a logic programming language. This means that the primary concept to deal with when programming in Bedelibry Prolog is that of a \textit{predicate}. In Bedelibry Prolog's type system, predicates have type \textit{pred}. 

If you have ever taken a course in formal logic, predicates in Bedelibry Prolog behave similarly to normal \textit{propositions} in classical logic. A predicate is some \textit{fact} that can either hold (in which case we say the predicate is \textit{true}), or not hold (in which case we say it is \textit{false}). To declare predicates in Bedelibry prolog (which always start with a lowercase letter), we use the syntax

\begin{Verbatim}[commandchars=\\\{\}]
  \textcolor{reservedColor}{rel} \textcolor{predColor}{p}.
\end{Verbatim}

This is a declaration that p is a \textit{relation} (which is the same thing as a predicate, except it can have an arbitrary number of arguments) with no arguments -- i.e. a \textit{proposition}. If a declaration like the above has been made in a Bedelibry Prolog file that has been loaded into the REPL with no other facts declared about p, then typing the query \begin{verbatim} p. \end{verbatim} into the command prompt at the REPL will yield:

\begin{verbatim}
  > No solutions found.
\end{verbatim}

This illustrates one important difference from classical logic in Bedelibry Prolog -- we make an \textit{open-world} assumption about facts by default. In other words, we have declared that \textit{p} is a proposition, and we have not yet delcared that \textit{p} is a \textit{fact}, but this does not mean that \textit{p} is false -- only that we cannot determine that \textit{p} is true. 

If we add a declaration \begin{verbatim} p. \end{verbatim} into one of our loaded files and we run the same query, we will instead get:

\begin{verbatim}
  > True.
\end{verbatim}

This is the basic type of query that we can make in Bedelibry Prolog -- some combination of predicates, relations and proposions, which can either be \textit{True}, or which we cannot determine to be true with our current state of knowledge.

The default argument of predicates and relations in Bedelibry Prolog is \begin{verbatim} entity \end{verbatim}. We can declare terms of type entity using the following syntax:

\begin{verbatim}
  x: entity.
\end{verbatim}

And relations taking 1 or more argument of type entity can be declared using the Prolog-inspired syntax:

\begin{Verbatim}[commandchars=\\\{\}]
  \textcolor{reservedColor}{rel} \textcolor{predColor}{p}/\textcolor{arityColor}{1}.
  \textcolor{reservedColor}{rel} \textcolor{predColor}{q}/\textcolor{arityColor}{2}.
\end{Verbatim}

So now, p(x), and q(x,x) are both valid Bedelibry Prolog terms under this schema.

\section{Facts and rules}

In Bedelibry Prolog, facts can be declared in a schema file with the same syntax with which we make queries at the REPL. For instance, if we have a schema file with the line

\begin{Verbatim}[commandchars=\\\{\}]
  \textcolor{predColor}{p}(\textcolor{entityColor}{x}).
\end{Verbatim}

\noindent then the query \textit{?- p(x).} at the REPL will yield \textit{True.} \medskip

\noindent Rules may also be defined in Bedelibry Prolog schemas. These have the following syntax:

\begin{Verbatim}[commandchars=\\\{\}]
  \textcolor{predColor}{a} :- \textcolor{predColor}{b}, \textcolor{predColor}{c}.
\end{Verbatim}

\noindent which roughly translates to ``a if b and c''. \medskip

\noindent To see how this works, consider the following example:

\begin{Verbatim}[commandchars=\\\{\}]
  \textcolor{reservedColor}{rel} \textcolor{predColor}{a}.
  \textcolor{reservedColor}{rel} \textcolor{predColor}{b}.
  \textcolor{reservedColor}{rel} \textcolor{predColor}{c}.

  \textcolor{predColor}{b}. \textcolor{predColor}{c}.
  \textcolor{predColor}{a} :- \textcolor{predColor}{b}, \textcolor{predColor}{c}.

  ?- \textcolor{predColor}{a}.
  > \textcolor{trueColor}{True}.
\end{Verbatim}

\noindent More interestingly, we can use \textit{variables} in rules, which in Bedelibry Prolog always start with a capital letter.

\begin{Verbatim}[commandchars=\\\{\}]
  \textcolor{reservedColor}{rel} \textcolor{predColor}{man}/\textcolor{arityColor}{1}.
  \textcolor{reservedColor}{rel} \textcolor{predColor}{mortal}/\textcolor{arityColor}{1}.
  
  \textcolor{predColor}{man}(\textcolor{entityColor}{socrates}).

  \textcolor{predColor}{mortal}(\textcolor{varColor}{X}) :- \textcolor{predColor}{man}(\textcolor{varColor}{X}).
  \end{Verbatim}

\noindent Free variables in Bedelibry Prolog are implicitly universally quantified, so the above rule reads as ``all men are mortal.'' And from the schema above, Bedelibry Prolog can deduce:

\begin{verbatim}
  mortal(socrates).
  > True
\end{verbatim}

\section{Types}

In the last two sections we learned about the basic type system in Bedelibry, and how to declare facts and rules. At this point, our system is essentially the same as vanilla Prolog -- except that there is a distinct seperation between \textit{predicates} and their arguments, which we must declare explicitly before use. However, we can use our typesystem for much more than this. For instance, in our ``all men are mortal'' example, instead of using the predicate \textit{man}, we can use a type for the same thing instead, which we declare with the syntax

\begin{verbatim}
  type man.
\end{verbatim}

We can then specify a more specific type for our predicate \textit{mortal}

\begin{Verbatim}[commandchars=\\\{\}]
  rel mortal: man.
\end{Verbatim}

which says that the predicate \textit{mortal} can only be meaningfully applied to terms of type \textit{man}. Thus, the following will not typecheck:

\begin{Verbatim}[commandchars=\\\{\}]
  type man.
  type chair.
  rel mortal: man.
  socrates: man.
  my_chair: chair.

  % The predicate "mortal" doesn't make sense for chairs.
  mortal(my_chair).
\end{Verbatim}

In addition, with the declaration that \textit{man} is a type, Bedelibry Prolog automatically generates a predicate (called a \textit{type predicate})

\begin{Verbatim}[commandchars=\\\{\}]
  \textcolor{reservedColor}{rel} \textcolor{predColor}{man}/\textcolor{arityColor}{1}.
\end{Verbatim}

which only returns True on predicates of type \textit{man}. Thus, we can write our rule for "all men are mortal" the same as before, but with the additional gaurntee now that the predicate \textit{mortal} is now only used in meaningful situations.

\subsection{Open and closed types/relations}

I mentioned earlier that in Bedelibry Prolog we make an \textit{open world assumption} on our predicates. This has two (related) effects:

\begin{enumerate}
  \item Rules and facts regarding an open relation may be defined in modules outside of where the relation was declared.
  \item We do not have the final say on the falsehood of queries involving an open relation.
\end{enumerate}

\noindent However, Bedelibry Prolog gives the option of marking relations as \textit{closed}, using the syntax:

\begin{Verbatim}[commandchars=\\\{\}]
  \textcolor{reservedColor}{closed rel} \textcolor{predColor}{p}.
\end{Verbatim}

\noindent This (correspondingly) has the effect that:

\begin{enumerate}
  \item Rules and facts regarding a closed relation may \textbf{not} be defined in modules outside of where the relation was defined.
  \item We \textbf{do} have the final say on the falsehood of queries involving a closed relation.
\end{enumerate}

\noindent Thus, we have:

\begin{Verbatim}[commandchars=\\\{\}]
  \textcolor{reservedColor}{closed rel} \textcolor{predColor}{p}.
  ?- \textcolor{predColor}{p}.
    > \textcolor{falseColor}{False}.
\end{Verbatim}

\noindent A similar semantics works in Bedelibry Prolog for \textit{closed types}. Entities of a closed type can only be declared in the same module the type was declared in, and the type predicate corresponding to a closed type is a closed relation.

\section{Assertions}
\epigraph{And is there also not the case where we play, and we make up the rules as we go along? And even where we alter them -- as we go along.}{Wittgenstein}

Although having a schema may be suitable for many purposes, Bedelibry Prolog also supports the ability to dynamically assert new predicates, rules, types, and entities. This can be done in the REPL by terminating a Bedelibry Prolog term with a \textit{!} instead of the usual \textit{.} .

\begin{verbatim}
  ?- p.
    > No solutions.
  ?- p!
  ?- p.
    > True.
\end{verbatim}

\section{Metainterpreters}

\epigraph{But what does a game look like that is everywhere bounded by rules? whose rules never let a doubt creep in, but stop up all the gaps where it might? -- Can't we imagine a rule regulating the application of a rule; and a doubt which it removes -- and so on?}{Wittgenstein}



\chapter{Functional Programming}

Functional programming and logic programming are both \textit{declarative} programming language paradigms. In contrast to \textit{imperative} programming langauges, where program are written by telling the computer \textit{how} to preform a certain computation, in declarative languages, we describe \textit{what} a computation is supposed to do, and leave the details of exactly \textit{how} such computations are executed to the compiler/interpreter. This allows us to write code at a higher-level of abstraction, requiring on average less code to accomplish the same task as if we were programming in an imperative language, however, has the disadvantage that it is harder to reason about preformance and efficency issues (note, however, that this does not mean that declarative languages can not be efficent!).

As anyone who has taken a course in discrete mathematics (or an introduction to mathematical logic) could tell you -- a function is a specail case of a relation, and thus, the functional and logical paradigms are very closely related. Language designers have taken advantage of this relationship to develop hybrid logical/functional languages such as \textit{Curry} and \textit{Mercury}, and Bedelibry Prolog also supports this mixture of paradigms.

Any relation in Bedelibry Prolog can be declared using the following ``functional'' (or ``equational'') syntax, which just serves to emphasize that we are thinking about a relation in a functional way: 

\begin{Verbatim}[commandchars=\\\{\}]
 \textcolor{commentColor}{% Functional syntax}
   \textcolor{predColor}{f}(\textcolor{entityColor}{x},\textcolor{entityColor}{y},\textcolor{entityColor}{z}) = \textcolor{entityColor}{w}.
 \textcolor{commentColor}{% The equivalent expression using the standard relational syntax.}
   \textcolor{predColor}{f}(\textcolor{entityColor}{x},\textcolor{entityColor}{y},\textcolor{entityColor}{z},\textcolor{entityColor}{w}).
\end{Verbatim}

\section{Algebraic data types}

While not all functional programming languages support it, a popular feature of statically typed functional programming languages is that of \textit{algebraic data types}, or ADTs. Algebraic data types work by supplying a finite number of different constuctors or \textit{variants} of that datatype. For instance, one of the simplest algebraic data types which can be defined in Bedelibry Prolog is \textit{bool}:

\begin{Verbatim}[commandchars=\\\{\}]
  \textcolor{reservedColor}{datatype} \textcolor{typeColor}{bool} where
    \textcolor{reservedColor}{constructor} \textcolor{constrColor}{'True}.
    \textcolor{reservedColor}{constructor} \textcolor{constrColor}{'False}.
\end{Verbatim}

\noindent This says that the type of booleans \textit{bool} consists of two constructors, \textit{'True} and \textit{'False}. So far, this feature can be seen as being similar to \textit{enums} in other programming languages. However, algebraic data types are more general, because their constructors can also have \textit{arguments}. For example, consider the following algebraic datatype, which might be used in a game with different types of \textit{agents}:

\begin{Verbatim}[commandchars=\\\{\}]
  \textcolor{reservedColor}{datatype} \textcolor{typeColor}{agent} where
    \textcolor{reservedColor}{constructor} \textcolor{constrColor}{'Person}: \textcolor{typeColor}{string}.
    \textcolor{reservedColor}{constructor} \textcolor{constrColor}{'Cat}.
    \textcolor{reservedColor}{constructor} \textcolor{constrColor}{'Dog}.
\end{Verbatim}

\noindent This says that an agent is either a \textit{'Person}, with an argument of type \textit{string} (presumably, a name), for example \textit{'Person("Larry")'}, a \textit{'Dog}, or a \textit{'Cat}.

Algebraic data types can also be recursive, giving us the ability to describe datastructures such as lists and trees:

\begin{Verbatim}[commandchars=\\\{\}]
  \textcolor{reservedColor}{datatype} \textcolor{typeColor}{intlist} where
    \textcolor{reservedColor}{constructor} \textcolor{constrColor}{'Nil}.
    \textcolor{reservedColor}{constructor} \textcolor{constrColor}{'Cons}: \textcolor{typeColor}{int}, \textcolor{typeColor}{intlist}.
\end{Verbatim}

\section{Aside: Datatypes v.s. Entities in Bedelibry}



\section{Structural types}

\section{Parametric Polymorphism}

\begin{Verbatim}[commandchars=\\\{\}]
  \textcolor{reservedColor}{datatype} \textcolor{typeColor}{list}[\textcolor{varColor}{A}] \textcolor{reservedColor}{where}
    \textcolor{reservedColor}{constructor} \textcolor{constrColor}{'Nil}.
    \textcolor{reservedColor}{constructor} \textcolor{constrColor}{'Cons}: \textcolor{typeColor}{int}, \textcolor{typeColor}{list}[\textcolor{varColor}{A}].
\end{Verbatim}

\chapter{Imperative and Object-Oriented Programming}

Thus far in our tutorial, we have only seen how to write Bedelibry Prolog files containing schema declarations, facts, and rules, and how to make queries in the Bedelibry Prolog REPL using this loaded data. In this chapter, we will discuss the imperative programming features of Bedelibry Prolog, and exlain how to use the Bedelibry Prolog compiler \textit{blic} to make standalone exectuables from \textit{.bpl} files. As per tradition, we'll start with "Hello, world!". Create a file named \textit{hello\textunderscore world.bpl} with the following contents:

\begin{Verbatim}[commandchars=\\\{\}]
\textcolor{reservedColor}{using} \textcolor{moduleColor}{system_io}.

?- \{
  print_ln(\textcolor{stringColor}{"Hello, world!"}).
  print_ln(\textcolor{stringColor}{"Witaj świecie!"}).
  print_ln(\textcolor{stringColor}{"こんにちは世界!"}).
\}
\end{Verbatim}

Running \textit{bplc hello\textunderscore world.bpl} will produce an executable in the same directory as the .bpl source file named \textit{hello\textunderscore world}. However, there are some syntatic features of Bedelibry Prolog which we now need to discuss to understand how this program works.

\textit{print\textunderscore ln} is a \textit{procedure} imported from the module \textit{system\textunderscore io} which prints it's argument to the screen. It is declared with the type declaration:

\begin{verbatim}
  proc print_ln: string.
\end{verbatim}

\textit{string} here is the type of string literals, such as \textit{"Hello, world!"}, \textit{"Witaj świecie!"}, and \textit{"こんにちは世界!"} (Bedelibry Prolog supports unicode out of the box!). 

Procedures in Bedelibry Prolog files must be put inside an \textit{execution environment}, which starts with a \textit{?-}, and is sourrounded by curly brackets. Commands run in an execution environment (which may be procedures or queries) behave similarly to those entered in at the REPL. 

\section{Imperative programming in Haskell}

\section{Bedelibry Prolog's Permission system}

\section{Inheritance in Bedelibry}



\chapter{Bedelibry}

\section{Syntax}

\epigraph{Our language can be seen as an ancient city: a maze of little streets and squares, of old and new houses, and of houses with additions from various periods; and this surrounded by a multitude of new boroughs with straight regular streets and uniform houses.}{Wittgenstein}

36. Where our language suggest a body and there is none: there, we should like to say, is a spirit

But how many kinds of sentence are there? Say assertion, qeustion, and command? -- There are \textit{countless} kinds; countless different kinds of use of all the things we call "signs", "words", "sentences". And this diversity is not something fixed, given once and for all, but new types of language, new language-games, as we might say, come into existence, and others become obsolete and get forgotten.

\section{Dealing with ambiguity}

\epigraph{And do we know any more ourselves? Is it just that we can't tell others exactly what a game is? -- But this is not ignorance. We dont know the boundaries because none have been drawn.}{Philosophical Investigations, remark 69.}

Much like in real life (i.e. in natural languages), terms in Bedelibry Prolog can be ambiguous. One of the basic ways in which this happens is because terms in Bedelibry prolog can be declared to be of two different types.

\begin{verbatim}
  type programming_language.
  type animal.

  python: programming_language.
  python <: animal.
\end{verbatim}

\noindent Usually context is sufficent to disambiguate the type of a Bedelibry Prolog term, but if this is not the case, then 

\subsection{Fuzzy concepts}

\epigraph{And likewise the kinds of number, for example, form a family. Why do we call something a "number"? Well, perhaps because it has a -- direct -- affinity with several things that have hitherto been called "number", and this can be said to give it an inndirect affinity with other things that we also call "numbers". And we extend our concept of number as in spinning a thread we twist fibre on fibre. And the strength of the thread resides not in the fact that some one fibre runs through its whole length, but in the overlapping of many fibres.}{Philosophical Investigation, remark 67.}

\subsection{Sense disambiguation}

\epigraph{Consider this example: if one says "Moses did not exist", this may mean various things... 79}{Philosophical Investigation, remark 67.}

\subsection{Lambek Types}

\epigraph{I reject the contention that an important theoretical difference exists between formal and natural languages.}{Richard Montague}

\subsection{Case study: Research notes}



\subsection{Case study: Random quotation generator}

fortune is a unix command-line utility which generates random epigrams from a preset set of epigrams stored on the user's filesystem. 

\begin{verbatim}
> fortune | cowsay                                                                                                                                                                                             ✹ ✭
 _____________________________________
/ You'll feel much better once you've \
\ given up hope.                      /
 -------------------------------------
        \   ^__^
         \  (oo)\_______
            (__)\       )\/\
                ||----w |
                ||     ||
\end{verbatim}

\noindent Charming! The user of fortune can of course cofigure these epigrams in various ways, and can add/remove different epigrams by editing the relevant files that fortune uses in its configuration.

To illustrate some of the features of Bedelibry prolog that we've introduced so far, let's make our own version of fortune, but one which instead prints random quotes from Wittgenstein which we have marked in Bedelibry that we like. We'll call our applicaton \textit{wittquote}.

\begin{verbatim}
/* LANGUAGE BedelibryMode */

%
% _*_ wittquote.bpl _*_
%

using ontology_literature.
using philosoper_quotes.
using system_io.
using control_random.

?- {
  quote(?X), written_by(?X, wittgenstein), likes(me, ?X).
  random_select(?X). 
  print_ln(?X).
}
\end{verbatim}

\end{document}